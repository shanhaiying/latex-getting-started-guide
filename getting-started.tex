%%%%% This is the preamble of this document
\documentclass{article}
\usepackage[nohead,margin=1.45in]{geometry}
\usepackage{amsthm}
\newtheorem{theorem}{Theorem}
\newtheorem*{theorem*}{Theorem}
\parskip=2pt
\title{Getting Started with \LaTeX\\in the Sagemath Cloud}
\author{S.R.~Doty}

%%%%%% This is the body of this document
\begin{document}
\maketitle

\noindent
\LaTeX\ is a system for creating beautiful documents containing complicated mathematics, tables, diagrams, images, and code.
To try it out without any commitment, open a modern browser (not Internet Explorer) and go to
\begin{center}
\texttt{cloud.sagemath.org.}\footnote{
The Sagemath Cloud\texttrademark\ is a website developed by William Stein of the University of Washington, supporting many types of computational mathematics (including \LaTeX, Python, and Sage) in the cloud using a Sage engine running in the background.}
\end{center}

\noindent Go ahead and sign up for a free account. Once you are logged in, create a new project and select the \LaTeX\ option. This opens a pair of side-by-side panes in the browser. The left pane contains \LaTeX\ commands and the right one shows the ``compiled'' paper.

There should already be a bare paper skeleton there. Make changes by typing in the left pane. Every time you hit the Save button, the right pane is updated. When you're done editing, click on the Files menu to download your work. If you click on Files now, you will see files with extensions of \texttt{synctex.gz}, \texttt{pdf}, \texttt{log}, \texttt{aux}, and \texttt{tex}. These are all created by \TeX's engine when your paper is compiled; the most important ones are the \texttt{tex} file (where the \LaTeX\ input is stored) and the \texttt{pdf} file for the converted output. If you have an existing \texttt{tex} file you can upload it to the project. Most people learn \LaTeX\ by altering someone else's existing document.

In the Sagemath Cloud, you can easily share your project with one or more collaborators, and they will be able to edit the document, too.

\LaTeX\ is a markup language. You type source commands in one file, which are instructions for what goes where in the document, and then \LaTeX\ converts the code into a PDF document.

Leslie Lamport created the original version of \LaTeX\, based on the earlier \TeX\ project, created in the 1980s by Donald Knuth. Both are open-source and free. You can think of \LaTeX\ as a layer of convenient macros placed on top of \TeX; in particular, when you use \LaTeX\ you are also using \TeX, and it is \TeX's engine that does most of the processing.

All \TeX\ commands begin with a backslash ($\backslash$). A \LaTeX\ file begins with a \verb!\documentclass! command and always contains a \verb!\begin{document}!---\verb!\end{document}! pair. Anything after \verb!\end{document}! is ignored. Between the \verb!\documentclass! first line and \verb!\begin{document}! is an area called the \emph{preamble}; this is where you put commands with global effects, set options, define your title and author, and other preliminary things. Between \verb!\begin{document}! and \verb!\end{document}! is the \emph{body} of the paper, where you type the paper's content. Anything typed after a \verb!%! symbol is a comment in the file, ignored by \TeX.

To write prose, type it in the body as usual in the left pane. One or more blank lines always indicates a new paragraph.
The command \verb!\noindent! before a paragraph suppresses the indentation if its first line.
Spacing is automatically calculated by the \TeX\ engine, so you may leave as many blank spaces between words as you like (\TeX\ ignores extra ``white space''). To \emph{emphasize} text use the command \verb!\emph{emphasize}!.  Bold text is achieved similarly by the \verb!\textbf! command, for instance typing \verb!\textbf{bold}! produces \textbf{bold}. The command \verb!\underline{underline}! produces the result \underline{underline}.

Two types of math are supported: inline and display. Inline math is always placed between a pair of dollar signs, while display math is placed between a pair of double dollar signs. For example, typing \verb!$x^2+y^2=z^2$! produces the inline math equation $x^2+y^2=z^2$ while typing \verb!$$x^2+y^2=z^2,$$! produces $$x^2+y^2=z^2,$$ a displayed version of the same equation. Displayed math appears centered on a line by itself. Other ways of producing display math exist. For example, putting math between \verb!\[! and \verb!\]! is equivalent to surrounding it by double dollar signs (so typing \verb!$$x^2+y^2=z^2$$! is equivalent to typing \verb!\[x^2+y^2=z^2\]!). White space in math mode is always ignored by \TeX\ since \TeX\ computes the spacing in math mode (but there are special commands for tweaking that when needed).

\LaTeX\ environments are used for special formatting needs. An environment always starts with \verb!\begin{<ename>}! and ends with \verb!\end{<ename>}!, where \verb!<ename>! is the name of the environment. For example, use the \verb!enumerate! environment to create an indented numbered list. Typing the input:
\begin{verbatim}
\begin{enumerate}
\item This is the first thing.
\item This is the second thing.
\item This is the third.
\end{enumerate}
\end{verbatim}
produces the output
\begin{enumerate}
\item This is the first thing.
\item This is the second thing.
\item This is the third.
\end{enumerate}
An unnumbered list (with bullets) is created similarly using the \verb!itemize! environment in the same way.

Theorems, propositions, lemmas, corollaries, definitions, and examples may be formatted using a special environment for that purpose. Such theorem-like environments must be defined in the preamble. For example, after putting the defining command
\begin{verbatim}
\newtheorem{theorem}{Theorem}
\end{verbatim}
in the preamble of this document, I can type something like this in the body
\begin{verbatim}
\begin{theorem}[Fermat]
There do not exist positive integers $x, y, z$ such that
$x^n+y^n=z^n$ when $n$ is any integer greater than 2.
\end{theorem}
\end{verbatim}
which is typeset by \LaTeX\ as follows:
\begin{theorem}[Fermat]
There do not exist positive integers $x, y, z$ such that
$x^n+y^n=z^n$ when $n$ is any integer greater than $2$.
\end{theorem}
To suppress the (automatic) theorem number, it is necessary to load the \verb!amsthm! package in the preamble and also alter the command defining the theorem environment, as follows:
\begin{verbatim}
\usepackage{amsthm}
\newtheorem*{theorem*}{Theorem}
\end{verbatim}
which should define an unnumbered theorem environment named \verb!theorem*!. Now typing the input:
\begin{verbatim}
\begin{theorem*}[Fermat]
For any integer $n$ greater than $2$, there do not
exist positive integers $x, y, z$ such that
$x^n+y^n=z^n$.
\end{theorem*}
\end{verbatim}
produces the output
\begin{theorem*}[Fermat]
For any integer $n$ greater than $2$, there do not
exist positive integers $x, y, z$ such that
$x^n+y^n=z^n$.
\end{theorem*}

There are many packages (such as \verb!amsthm!) which enhance the capabilities of \LaTeX. Use the \verb!geometry! package to control the margins of the resulting document. Use the \verb!amsmath! and \verb!amssymb! packages to get various convenient extra environments for display math and extra special symbols, respectively. For instance, from amsmath you get the useful \verb!align! and \verb!align*! environments, for creating sequences of aligned displayed equations; this is useful for displaying mathematical derivations. 

Computer code is easily rendered in a monospaced font so as to preserved all spacing and endlines, by placing it in a \verb!verbatim! environment. The following Python code was typeset by placing it within \verb!\begin{verbatim}! and \verb!\end{verbatim}!.
\begin{verbatim}
def hello_world(n):
    """Say hello n times"""
    for j in range(n):
        print("Hello, world!")
\end{verbatim}
As you can see, the important indentation was preserved. For fancier ways to display computer code in \LaTeX\ check out the \verb!listings! package, which provides the \verb!lstlisting! environment.

To get started, grab an existing \LaTeX\ file, upload it to the Sagemath Cloud, and start experimenting. You see the effect of your changes as you make them, and you can't break anything, so be brave. 

Google search will quickly find answers to most \LaTeX\ questions. In particular, search for \LaTeX\ cheat sheets, which can be very helpful.
There are useful online books about \LaTeX; one of the better ones is the \emph{Not so Short Introduction to} \LaTeX\ by Tobius Oetiker et al. Printed books are also available if you want to really learn all the myriad ins--and--outs; one of the best is the book \emph{Math into} \LaTeX\ by George Gr\"{a}tzer.

Happy \LaTeX-ing!
\end{document}

%sagemathcloud={"zoom_width":145}
